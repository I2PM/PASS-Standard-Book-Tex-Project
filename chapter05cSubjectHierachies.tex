\newpage
\section{Hierarchies in Communication Oriented Business Process Models}

PASS  offers powerful possibilities for structuring complex process systems. The ways to do that are demonstrated with an example.
As an example we will consider a process for realizing a car break down service. This service consists of several connected processes. There is the main process for handling the car accident and supporting e.g. processes for organising towing and repair shop services. Insurance companies may be involved for covering damages, the customer gets an invoice, uses money transfer services or banks for paying the invoice. These processes are executed by various organisations like help desk service companies, towing service companies, car repair workshops banks etc.. In most business process projects not all parts of processes are described in detail. Only a certain part is considered, e.g. only the help desk process has to be considered in detail. In order to do so we have to consider the whole environment in which a considered process is embedded. We have to know which relations exists to these other processes. It is necessary to know which inputs are rquired by neighbour processes and which results they deliver. A help desk process which organizes the towing services has to know how the towing service is requested and which further interactions are required. For instance it must be agreed whether the towing service informs the client about the arrival time of the towing truck or the help desk does it.


\subsection{Process Architecture}

Rectangles represent processes. Each process has a name. Processes consists of other processes and/or subjects. The lines between the rectangles represent the communication channels between processes. Each communication channel has a nameand can contain other communication chan-nels and/or messages.

Figure \ref{fig:car-service-level1} shows the highest process level of the car break down service. In the "car use" process the event "car break down" happens. In order to organize support an interaction is initiated with process "car break down service" . Between these processes messages are exchanged which are elements of the communication channel "Car break down handling".\\


\begin{figure}[htbp]
	\centering
	\includegraphics[width=0.7\linewidth]{Figures/Chapter5/figures-hierarchy/Car-Service-Level1.jpg}
	\caption[High level structure of car break down service]{High level structure of car break down service}
	\label{fig:car-service-level1}
\end{figure}



Figure \ref{fig:car-service-leve2} shows the next process structure level of the process "car break down ser-vice". In this level the process "Car break down service" is spltid in 10 processes. The processes "Bank", "Insurance service", "Car repair workshop", "Incident Management","Mobility Manage-ment" and "Towing Management" have a communication channel to the prcess "Car usage". This means the communication channel "Car break down handling" is split into five communica-tion channels. Each of them covers the communication with the relared process, e.g. the communi-cation channel "Accident notification Car break down" is the communication channel between the processes "Car usage" and "Incident Management".\\


\begin{figure*}[htbp]
	\centering
	\includegraphics[width=0.8\linewidth]{Figures/Chapter5/figures-hierarchy/Car-Service-Leve2}
	\caption[Structure of the Emmergency Call Handling Process]{Structure of the Emmergency Call Handling Process}
	\label{fig:car-service-leve2}
\end{figure*}



Inside a process there can be also processes. This means that levels of processes can be built. Figure \ref{fig:car-service-lev3} shows the next deeper level of our process hierarchy. The process "Car repair workshop" is structured in six processes. According to this separation the communication sets are also splitted e.g. the communication set "Handling repair service" is splitted into three parts, one part is han-dled by the process "Service scheduling" the other by the process "Car droping" and the third one by the process "Customer Satisfaction".\\

\begin{figure*}[htbp]
	\centering
	\includegraphics[width=0.8\linewidth]{Figures/Chapter5/figures-hierarchy/Car-Service-Lev3}
	\caption[Details of the "Car repair workshop" Process]{Details of the "Car repair workshop" Process}
	\label{fig:car-service-lev3}
\end{figure*}

As already mentioned, processes cannot communicate directly with each other. The active entities of a process, the subjects communicate with each other. This means messages from one process are sent to an other process are reveived by a subject inside of that process. Messages belonging to a channel are assigned to a sending or receiving subject at the lowest level of a process architecture. This lowest level of a process description is the subject interaction diagram (SID) which shows the involved subjects of a process and the messages they exchange. In the following we consider the process incident management in more detail. This process does not contain other processes like the process "Car Repair Shop". The process "Incident management" contains a Subject Interaction Diagram. Some of the subjects of a process communicate with subjects in other processes. These subjects are called border subjects because they are at the border of a process to other prcesses. Figure \ref{fig:car-service-lev4} shows the process "Incident management" with its border subjects. There is a border subject "Help agent" which communicates with the processes "Towing service", "Mobility ser-vice"and  "Car repair workshop", precisely it communicates with a subject in one of these processes. Another border subject of the process "Incident management" which is called "Help desk"communicates with the process "Car usage".\\

\begin{figure}[htbp]
	\centering
	\includegraphics[width=0.9\linewidth]{Figures/Chapter5/figures-hierarchy/Car-Service-Lev4}
	\caption[Neighbors of the "Incident Manaement Process"]{Neighbors of the "Incident Manaement Process"}
	\label{fig:car-service-lev4}
\end{figure}

The border subjects of the process "Incident management" must have a coresponding border sub-ject at the neighbour processes. The border subjects "Call agent" communicates with the border subject "Help requestor" of process "Car usage" and the border subject "Help agent" communi-cates with border subjects of the processes "Car repair workshop", "Towing service and "Mobility service". The process "Incident management" with all the border subjects is shown in  figure \ref{fig:car-service-lev5}.\\

\begin{figure}[htbp]
	\centering
	\includegraphics[width=0.9\linewidth]{Figures/Chapter5/figures-hierarchy/Car-Service-Lev5}
	\caption[Border subjects of the "Incident Management" Process]{Border subjects of the "Incident Management" Process}
	\label{fig:car-service-lev5}
\end{figure}

The border subjects of the processes "Mobility service", "Towing service" and "Car repair work-shop" have the same name “Service agent” but these are different subjets because they belong to different processes. Because the process "Car repair workshop" consists of several layers the corre-sponding border subject can be in a process which is part of process "Car repair workshop" in a lower level.\\
From the perspective of the subjects inside of the process "Incident managent" are the border subjects of the processes "Mobility service", "Towing service" and "Car repair workshop" interfaces to these processes, therefore they are called interface subjects in the Subject Interaction Diegram of a process. Figure \ref{fig:car-service-lev6}shows the Subject Interaction Diagram of the process incident management.\\


\subsection{Behavioral Interface}
Processes to which a considered process has communication relationships are called process neighbours or for short neighbours. Now we want to consider the details of the communication relationships between two neighbours. The interface between two processes is defined by the related border subjects and the allowed sequences in which the messages are exchanged between them in a communication channel. As already described above each message is defined by a name and the data which are transported the so-called payload. A border subject observes the behavior of the border subject of the neighbour process and vice versa. Figure \ref{fig:car-service-lev8} shows the border subject "Help desk" of the processes "Incident Management" which communicates with the border subject of process "Car usage".\\

\begin{figure}[htbp]
	\centering
	\includegraphics[width=1.0\linewidth]{Figures/Chapter5/figures-hierarchy/Car-Service-Lev6}
	\caption[Subject Interaction Diagram of the Process "Incident Management"]{Subject Interaction Diagram of the Process "Incident Management"}
	\label{fig:car-service-lev6}
\end{figure}

Because we consider the process "Incident management" the border subject "Caller" of the process "Car usage" becomes an interface subject in the SID (details about interface subjects can be found in \cite{Flei12}) of the process “Incident Management”. Figure \ref{fig:car-service-lev8} shows the detailed Subject Interaction Diagram around the subject help desk. \\

\begin{figure}[htbp]
	\centering
	\includegraphics[width=0.9\linewidth]{Figures/Chapter5/figures-hierarchy/Car-Service-Lev8}
	\caption[Subject Interaction around the subject “Help desk”]{Subject Interaction around the subject “Help desk”}
	\label{fig:car-service-lev8}
\end{figure}

Instead of the channels the messages required for a towing service request are shown. A message "Request towing service" comes from the interface subject. This message is accepted by the subject "help desk". The subject help desk checks the customer data received with this message by sending a corresponding the message "Get customer data" to the subject "Customer data management". This subject send the complete customer data back to the subject "Help desk" via the message "Customer data". The subject "Help desk" checks the customer data. If the data are invalid a message "Invalid customer data" is sent to the subject "Caller" and the process is finished.\
If the customer data are valid with that data the subject "Help desk" creates a trouble ticket which is sent to the subject "Ticket management". After that the message "Towing service requested" is sent to the help agent which organizes the towing service. The part of the communication structure of the subject "Help agent" in order to organize the towing service is not shown in figure \ref{fig:car-service-lev9}. We only see that subject "Help agent" sends the message "Towing service data" to the subject "Help desk". This message contains all the data about the service e.g. name of the towing company and arrival time. The subject "Help desk" forwards that data to the interface subject "Caller". This behavior is shown in figure \ref{fig:car-service-lev8}.\\



\begin{figure}[htbp]
	\centering
	\includegraphics[width=0.7\linewidth]{Figures/Chapter5/figures-hierarchy/Car-Service-Lev9}
	\caption[Part of the Behavior Diagram of the subject “Help desk”]{Part of the Behavior Diagram of the subject “Help desk”}
	\label{fig:car-service-lev9}
\end{figure}


The behavior described in the figure above contains the communication with all neighbor subjects of subject "Help desk" including the communication with the interface subject "Caller". From the perspective of this subject the communication of the subject "Help desk" with its other neighbor subjects is not relevant. For the subject "Caller" only the commumication sequence between itself and the subject "Help desk" is relevant. These allowed communication sequences are called the behavioral interface.\\
The behavioral interface between two subjects can be derived from the complete behavior of a subject by deleting the interactions with all the other subjects . Figure \ref{fig:car-service-lev10} shows how the communication sequence relevant for the communication be-tween the subject "Help Desk" and "Caller" is derived from the complete behavior of subject "Help desk".

\begin{figure}[htbp]
	\centering
	\includegraphics[width=1.0\linewidth]{Figures/Chapter5/figures-hierarchy/Car-Service-Lev10}
	\caption[Deriving the Behavioral Interface from the Subject Behavior]{Deriving the Behavioral Interface from the Subject Behavior}
	\label{fig:car-service-lev10}
\end{figure}

A behavoral interface is always relative to a communication partner. In figure \ref{fig:car-service-lev10} the behavioral interface is relative to the interface subject "Caller". The behavioral interface to the subject "Ticket Management" is different because only the communication activities with this subject are considered.This behavioral interface would be very simple. It consists of only one send activity, sending the message "Store ticket".\\
The behavioral interface relative to a partner subject can be automatically derived from the complete behavior of a subject
(see \cite{article:jCPEX}).

\subsection{Future Work}

Due to the novel conceptual integration addressed, several aspects and topics need to be addressed by future research:
\begin{list}{-}{spacing}
	\item Clarify terminology e.g. using the term interface subject, system interface, implementation
	\item Definition of structural semantics in OWL
	\item Definition of execution semantics in ASM. The semantic of the behaiour interface and its relation to the behavior of the related subject has to be described.
\end{list}