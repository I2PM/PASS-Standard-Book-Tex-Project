The following tables show the relationships between the PASS ontology and the PASS execution semantics described as ASMs.
Because of historical reasons in the ASMs names for entities and relations are different from the names used in the ontology.
The tables below show the mapping of the entity and relation names in the ontology to the names used in the ASMs.

\section{Mapping of ASM Places to OWL Entities}
Places are formally also functions or rules, but are used in principle as passive/static storage places.

%appendix in smaller font size
\footnotesize

\begin{landscape}
	\begin {longtable} {| p{0.5\textwidth} | p{0.3\textwidth} | p{0.6\textwidth}|}
	\hline
	OWL Model element &   ASM interpreter & Description\\
	\toprule
	\endhead
	\hline
	
	X - Execution concept – the state the subject is currently in as defined by a \textbf{State} in the model 
	& \textit{SID\_state} 
	&  Execution concept – no model representation, Not to be confused by a model “state” in an SBD Diagram. State in the SBD diagram define possible SID\_States.
	\\
	\hline
	
	\textbf{SubjectBehavior }	– under the assumption that it is complete and sound.
	& \textit{D} 
	&  A Diagram that is a completely connected SBD
	\\
	\hline
	
	\textbf{State}
	& \textit{node} 
	&  A specific element of diagram D -	Every node 1:1 to state
	\\
	\hline
	
	\textbf{State}
	& \textit{state} 
	& The current active state of a diagram determined by the nodes of Diagram D
	\\
	\hline
	
	\textbf{InitialStateOfBehavior, \newline EndState }
	& \textit{initial state, \newline end state} 
	& The interpreter expects and SBD Graph D to contain exactly one initial (start) state and at least one end state.
	\\
	\hline
	
	\textbf{Transition}
	& \textit{edge / outEdge} 
	& “Passive Element” of an edge in an SBD-graph
	\\
	\hline
	
	\textbf{TransitionConditionn}
	& \textit{ExitCondition} 
	& Static Concept that represents a Data condition
	\\
	\hline
	
	Execution Concept – ID of a Subject Carrier responsible possible multiple Instances of according to specific  \textbf{SubjectBehavior}
	& \textit{subj} 
	& Identifier for a specific Subject Carrier that may be responsible for multiple Subjects
	\\
	\hline
	
	Represented in the model with \textbf{InterfaceSubject}
	& \textit{ExternalSubject} 
	& A representation of a service execution entity outside of the boundaries of the interpreter
	(The PASS-OWL Standardization community decided on the new Term of Interface Subject to replace the often-misleading older term of External Subject)
	
	\\
	\hline
	\textbf{SubjectBehavior} or rather \textbf{SubjectBaseBehavior} as MacroBehaviors and GuardBehaviors are not covered by Börger
	& \textit{subject-SBD / \newline SBDsubject\textsubscript{subject}} 
	& Names for completely connected graphs / diagrams representing SBDs
	\\
	\hline
	
	Object Property: \textbf{\textit{hasFunctionSpecification}} 
	(linking \textbf{State}, and \textbf{FunctionSpecification} -->(\textbf{State} \textbf{\textit{hasFunctionSpecification }} \textbf{FunctionSpecification})
	& \textit{service(state) / \newline service(node)} 
	& Rule/Function that reads/returns the service of function of a given state/node
	\\
	\hline
	
	\textbf{DoState} \newline \textbf{SendState} \newline \textbf{ReceiveState}
	& \textit{function state, \newline send state, \newline receive state} 
	& The ASM spec does not itself contain these terms. The description text, however, uses them to describe states with an according service (e.g. a state in which a (ComAct = Send) service is executed is referred to as a send state)
	Seen from the other side: a SendState is a state with service(state) = Send)
	\newline
	Both send and receive services are a ComAct service.
	The ComAct service is used to define common rules of these communication services.
	
	\\
	\hline
	\textbf{CommunicationActs} with sub-classes (\textbf{ReceiveFunction SendFunction}) \newline
	\textbf{\textit{DefaultFunctionReceive1\_EnvironmentChoice \newline
			DefaultFunctionReceive2\_AutoReceiveEarliest \newline
			DefaultFunctionSend }}
	& \textit{ComAct} 
	& Specialized version of Perform-ASM Rule for communication, either send or receive. These rules distinguish internally between send and receive.
	\\
	\hline
	
	
\end{longtable}
\end {landscape}

\section{Main Execution/Interpreting Rules}
The interpreter ASM Spec has main-function or rules that are being executed while interpreted.

\begin{itemize}
	\item BEHAVIOR(subj,state)
	\item PROCEED(subj,service(state),state)
	\item PERFORM(subj,service(state),state)
	\item START (subj,X, node)
\end{itemize}

These make up the main interpreter algorithm for PASS SBDs and therefore have no corresponding model elements but rather are or contain the instructions of how to interpret a model.





\begin{landscape}
	\begin {longtable} {| p{0.7\textwidth} | p{0.3\textwidth} | p{0.4\textwidth}|}
	\hline
	OWL Model element &   ASM interpreter & Description\\
	\toprule
	\endhead
	\hline
	
	Execution concept
	& \textit{BEHAVIOR(subj;state)} 
	&  Main interpreter ASM-rule/Method
	\\
	\hline
	
		Execution concept
	& \textit{BEHAVIOR(subj;node)}
	& ASM-Rule to interpret a specific node of Diagram D for a specific subject
	\\
	\hline
	
		Execution concept
	& \textbf{Behaviorsubj (D)}
	&  Set of all ASM rules to interprete all nodes/states in a SBD(iagram) D for a given subj (set of all \textit{BEHAVIOR(subj;node)}
	\\
	\hline
	
		\textcolor{orange}{\textbf{State}} \textcolor{blue}{\textbf{hasFunctionSpecification}} \textcolor{orange}{\textbf{FunctionSpecification}}
		\newline
		Specialized in:
			\newline
		\textcolor{orange}{\textbf{DoFunction}} and
			\newline
		\textcolor{orange}{\textbf{CommunicationActs}} with
			\newline
		\textcolor{orange}{\textbf{ReceiveFunction}}
			\newline
		\textcolor{orange}{\textbf{SendFunction}}
			\newline
		There exist a few default activities:
			\newline
		\textcolor{purple}{\textbf{DefaultFunctionDo1\_EnvoironmentChoice}}
			\newline
		\textcolor{purple}{\textbf{DefaultFunctionDo2\_AutomaticEvaluation}}
	& \textit{PERFORM(subj ; service(state); state)}
	&  Main interpreter ASM-rule/Method
	\\
	\hline
	
	\textcolor{orange}{\textbf{CommunicationActs}} with
	\newline
	\textcolor{orange}{\textbf{ReceiveFunction}}
	\newline
	\textcolor{orange}{\textbf{SendFunction}}
	\newline
	\textcolor{purple}{\textbf{DefaultFunctionReceive1\_EnvironmentChoice}}
	\newline
	\textcolor{purple}{\textbf{DefaultFunctionReceive2\_AutoReceiveEarliest}}
	\newline
	\textcolor{purple}{\textbf{DefaultFunctionSend}}
	& \textit{PERFORM(subj ;ComAct; state)}
	&  ASM-Rule specifying the execution of a Comunication act in an according state)
	\\
	\hline
	
\caption{Main Execution/Interpreting Rules}
\label{tab:places-Entities}
\end{longtable}
\end {landscape}


\section{Functions}

Functions return some element. They are activities that can be performed to determine something.
Dynamic functions can be considered as “variables” known from programming languages, they can be read and written.
Static functions are initialized before the execution, they can only be read.
Derived functions "evaluate” other functions, they can only be read. “They may be thought of as a global method with read-only variables” 

\begin{landscape}
	\begin {longtable} {| p{0.7\textwidth} | p{0.3\textwidth} | p{0.4\textwidth}|}
	\hline
	OWL Model element &   ASM interpreter & Description\\
	\toprule
	\endhead
	\hline
	
	Function that the return state should correspond to/be derived from one of the multiple \textcolor{orange}{\textbf{State }}in an SBD model
	& \textit{SID\_state(subj)} 
	&  Dynamic ASM-Function that stores the current state of a subj
	\\
	\hline
	
	\textcolor{orange}{\textbf{State }} \textcolor{blue}{\textbf{hasOutgoingTransition }} \textcolor{orange}{\textbf{Transition }}  \newline
	(input / worked on link  / output (Set of Transition)
	(linking State with )
	& \textit{OutEdge(state) \newline
		OutEdge(state;i)}
	& Function that returns the set of outgoing edges of a state or a single specific edge i
	\\
	\hline
	
	Object Property: \textcolor{blue}{\textbf{hasTargetState}}
	(linking \textcolor{orange}{\textbf{Transition}} and \textcolor{orange}{\textbf{State}} --->\newline
	 \textcolor{orange}{\textbf{Transition  }}  \textcolor{blue}{\textbf{hasTargetState }} \textcolor{orange}{\textbf{State}}
	& \textit{target(edge)/ \newline
	target(outEdge)} /
	&  Function that returns the follow up state of an outgoing transition (\textit{outEdge} is a special denomination for an \textit{edge}returned by the \textit{outEdge}-Function)
	\\
	\hline
	
	Object Property: \textcolor{blue}{\textbf{hasSourceState}}
	(linking \textcolor{orange}{\textbf{Transition}} and \textcolor{orange}{\textbf{State}}-->
	\textcolor{orange}{\textbf{Transition}} \textcolor{blue}{\textbf{hasSourceState}} \textcolor{orange}{\textbf{State}}
	(input / worked on link  / output)
	& \textit{source(edge)}
	&  Function that returns the source state of an edge
	\\
	\hline
	\multicolumn{3}{c|}{\textbf{Determine Follow up state Mechanic}}
%	\\
%	\hline
%	eee& ZZZZ & nnnn
	\\
	\hline
	Exit conditions in PASS are defined on their corresponding
	\newline 
	\textcolor{orange}{\textbf{Transitions}} and therefore are called
	\newline
	\textcolor{orange}{\textbf{TransitionCondition}}
	\newline 
	\textcolor{orange}{\textbf{Transitions}}  have  (\textcolor{blue}{\textbf{hasTransitionCondition}})
	(\textcolor{orange}{\textbf{State}} $\rightarrow$
	\textcolor{blue}{\textbf{hasOutgoingTransition}} $\rightarrow$ \textcolor{orange}{\textbf{Transition }}  $\rightarrow$ \textcolor{blue}{\textbf{hasTransitionCondition}} $\rightarrow$ \textcolor{orange}{\textbf{TransitionCondition}})
	& \textit{ExitCond(e)}
	\newline
	\textit{ExitCond(outEdge)}
	\newline
	\textit{ExitCond\_i(e)}
	\newline
	\textit{ExitCond(e)(subj,state)}
	&  Derived Function that evaluates the ExitCondition of a given edge/outgoing edge 
	\\
	\hline
	Execution Concept 
	& select\textsubscript{Edge}
	&  ASM Function that determines an edged (transition) to follow.
	\\
	\hline
	Execution Concept (connected to: \textcolor{orange}{\textbf{State}}, and \textcolor{orange}{\textbf{FunctionSpecification}})
	& \textit{completed(subj ; \newline service(state); state)}
	&  Function that returns true if the Service of a certain state is complete IF the subject is in that state
	\\
	\hline
	Execution Concept
	& 
	&  Rule/Function that gives that returns the service of function of a given state
	\\
	\hline

	\caption{Derived Functions}
	\label{tab:Derived-Functions}
\end{longtable}
\end {landscape}

\section{Extended Concepts  – Refinements for the Semantics of Core Actions}

\begin{landscape}
	\begin {longtable} {| p{0.7\textwidth} | p{0.3\textwidth} | p{0.4\textwidth}|}
	\hline
	OWL Model element &   ASM interpreter & Description\\
	\toprule
	\endhead
	\hline
	
	Function that the return state should correspond to/be derived from one of the multiple \textcolor{orange}{\textbf{State }}in an SBD model
	& \textit{SID\_state(subj)} 
	&  Dynamic ASM-Function that stores the current state of a subj
	\\
	\hline
	
	\textcolor{orange}{\textbf{State }} \textcolor{blue}{\textbf{hasOutgoingTransition }} \textcolor{orange}{\textbf{Transition }}  \newline
	(input / worked on link  / output (Set of Transition)
	(linking State with )
	& \textit{OutEdge(state) \newline
		OutEdge(state;i)}
	& Function that returns the set of outgoing edges of a state or a single specific edge i
	\\
	\hline
	\caption{Refinrments places}
\label{tab:Refinrments places}
\end{longtable}
\end {landscape}


\section{Input Pool Handling}


\begin{landscape}
	\begin {longtable} {| p{0.7\textwidth} | p{0.3\textwidth} | p{0.4\textwidth}|}
	\hline
	OWL Model element &   ASM interpreter & Description\\
	\toprule
	\endhead
	\hline
	
	Refers to a set of \textcolor{orange}{\textbf{InputPoolConstraints }} of \textcolor{orange}{\textbf{Subject  }}that has \textcolor{blue}{\textbf{hasInputPoolConstraints}} – for its Input Pool  
	& \textit{constraintTable(inputPool )} 
	&  Function that Returns the set of all input Pool constrains
	\\
	\hline
	
	Execution Concept with evalution relevance for: \textcolor{orange}{\textbf{MessageSenderTypeConstraint}} and
	\textcolor{orange}{\textbf{SenderTypeConstraint }}
	& \textit{sender/receiver}
	& Identifiers for possible subject instances trying to access an input pool
	\\
	\hline
	
		Refers to a set of \textcolor{orange}{\textbf{InputPoolConstraints }} of \textcolor{orange}{\textbf{Subject  }}that has \textcolor{blue}{\textbf{hasInputPoolConstraints}} – for its Input Pool  
	& \textit{msgType )} 
	&  Function that Returns the set of all input Pool constrains
	\\
	\hline
	
	Execution Concept
	& \textit{select \textsubscript{MsgKind(subj ;state;alt;i)}} 
	&  ASM Function that determines the message kind (“message type”) to be received in a given receive state.
	\\
	\hline
	
	\textcolor{orange}{\textbf{InputPoolContstraintHandlingStrategy  }}
	\newline
	And their individual defaultinstances:
	\newline
	\textcolor{purple}{\textbf{InputPoolConstraintStrategy-Blocking }}
	\newline
	\textcolor{purple}{\textbf{InputPoolConstraintStrategy-DeleteLatest  }}
	\newline
	\textcolor{purple}{\textbf{InputPoolConstraintStrategy-DeleteOldest }}
	\newline
	\textcolor{purple}{\textbf{InputPoolConstraintStrategy-Drop}}
	& \textit{/{Blocking; DropYoungest; DropOldest; DropIncoming/}} 
	&  Default Input Pool handling strategies for 
	\\
	\hline
	
	Execution Concept – can be restricted by  \textcolor{orange}{\textbf{InputPoolConstraint}} – for its Input Pool  
	& \textit{P / inputPool} 
	&  The actual Input Pool
	\\
	\hline
	
	
	synchronous communication
	& \multicolumn{2}{p{9 cm}|}{Definition for an input pool \textbf{constraint set to 0} requiring sender and receiver interpreter to be in the corresponding send and receive states at the same time in order to actually communicate (as messages cannot be passed to an input pool)}
	\\
	\hline
	
	\caption{Input Pool Handling}
	\label{tab:Input-Pool-Handling}
\end{longtable}
\end{landscape}

\section{Other Functions}
%
\begin{landscape}
	\begin {longtable} {| p{0.6\textwidth} | p{0.4\textwidth} | p{0.4\textwidth}|}
	\hline
	OWL Model element &   ASM interpreter & Description\\
	\toprule
	\endhead
	\hline
	
	Exit conditions in PASS are defined on their corresponding \textcolor{orange}{\textbf{Transitions }} and therefore are called \textcolor{orange}{\textbf{TransitionCondition}}. Execution Concept: can be set on.
	Execution Concept: used to determine the correct exit
	& \textit{NormalExitCond} 
	&  is used internally to “remember” that neither a timeout nor a user cancel have happened, so that the correct exit transition can be taken.
	\\
	\hline
	In the model to be interpreted the according aspects are captured by \textcolor{orange}{\textbf{TimerTransitions}} that have (\textcolor{blue}{\textbf{hasTransitionCondition}}) a  
	\textcolor{orange}{\textbf{TimerTransitionCondition}} containing the date. The timeout(state) function should read the information. 
	&\multicolumn{2}{|p{11 cm}|}{\textbf{Timer/Timeout Mechanic:} The evaluation and handling of timeouts is defined (and refined) with several rules and functions.\textit{OutEdge(timeout(state), Timeout(subj , state, timeout(state)), SetTimeoutClock(subj ; state)}  are used to evaluate the timeout condition, \textit{OutEdge(Interrupt\_service(state)(subj , state)}  is used to define how the corresponding service should be canceled. \textit{OutEdge(TimeoutExitCond)}  is used internally to “remember” that a timeout happened, so that the correct exit transition can be taken.}
	

	\\
	\hline
	In PASS models the possibility to arbitrarily cancel the execution of a (receive) function and the possible course of action afterwards may be discerned via a \textcolor{orange}{\textbf{UserCancelTransitions }}
	&\multicolumn{2}{|p{11 cm}|}{\textbf{User Cancel/Abrupt Mechanic:} The evaluation and handling of user cancels is defined (and refined) with several rules and functions. \textit{UserAbruption(subj, state)} is used to evaluate the user decision, \textit{Abrupt\_service(state)(subj , state)} is used to define how the corresponding service should be abrupted. \textit{AbruptionExitCond} is used internally to “remember” that a user cancel happened, so that the correct exit transition can be taken.}
	\\
	\hline
	With the definition of the data properties \textcolor{green}{\textbf{hasMaximumSubjectInstanceRestriction}}
	The \textcolor{orange}{\textbf{}MultiSubject } are actually the standard and \textcolor{orange}{\textbf{SingleSubject}} the special case
	& \textit{MultiRound / mult(alt) / InitializeMultiRoun /
		ContinueMultiRoundSuccess 
		(among others}
	& Definition of Functions and ASM rules for interaction between multiple Subjects at once
		\\
	\hline
	
	Handling of \textcolor{orange}{\textbf{ChoiceSegment }} \& \textcolor{orange}{\textbf{	ChoiceSegmentPath}}

	 \textcolor{blue}{\textbf{hasOutgoingTransition }} \textcolor{orange}{\textbf{Transition }}  \newline
	(input / worked on link  / output (Set of Transition)
	(linking State with )
	& \textit{AltAction /
		altEntry(D) / altExit(D)
		AltBehDgm(altSplit)
		altJoin(altSplit)}
	& Rules for the semantics/handling of ChoiceSegements 
		\\
	\hline
	\textcolor{orange}{\textbf{State }} \textcolor{blue}{\textbf{hasOutgoingTransition }} \textcolor{orange}{\textbf{Transition }}  \newline
	(input / worked on link  / output (Set of Transition)
	(linking State with )
	& \textit{Compulsory(altEntry(D))} and textit{Compulsory(altExit(D))}
	& 
	\\
	\hline
	\caption{Other Functions}
	\label{tab:Other-Functions}
\end{longtable}
\end{landscape}
%
\section{Elements Not Covered not by Börger (directly)}
\begin{landscape}
	\begin {longtable} {| p{0.5\textwidth} | p{0.9\textwidth} | }
	\hline
	OWL Model element &    Description\\
	\toprule
	\endhead
	\hline
\textcolor{orange}{\textbf{ReminderTransition / ReminderEventTransitionCondition  }}
&This type time-logic-based transitions did not exist when the original ASM interpreter was conceived. They were added to PASS for the OWL Standard. They can be handled by assuming the existence of an implicit calendar subject that sends an interrupt message (reminder) upon a time condition (e.g. reaching of a calendarial date) has been achieved.
(includes the specialized (\textcolor{orange}{\textbf{CalendarBasedReminderTransition, TimeBasedReminderTransition}}

\\

	\hline
\textcolor{orange}{\textbf{DataDescribingComponent / DataMappingFunction }}
&The PASS OWL standard envisions the integration and usage of classic data element (Data Objects) as part of a process model. The Börger Interpreter does not assume the existence of such data elements as part of the model. However, the refinement concept of ASMs could easily been used to integrate according interpretation aspects.
(Includes Elements such as \textcolor{orange}{\textbf{PayloadDescription }} for Messages or \textcolor{orange}{\textbf{DataMappingFunction }}

\\
	\hline
\caption{Other Functions}
\label{tab:In ASM not covered}
\end{longtable}
\end{landscape}
%\textcolor{orange}{\textbf{Objekte}}\\
%\textcolor{blue}{Object properties}
%\textcolor{purple}{\textbf{constraint}}
%\textcolor{green}{data properties}
